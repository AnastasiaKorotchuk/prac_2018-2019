\documentclass{beamer}
\usepackage[utf8]{inputenc}
\usepackage[russianb]{babel}
\usepackage{amsmath}
\usetheme{Pittsburgh}
\usecolortheme{seahorse}

\begin{document}
	\title{Разведывательный анализ данных}
	\author{Арбузов П.А. \\ Кюнченкова Д.Д. \\ Семенов А.В.}
	\institute{312 группа}
	\date{2019}
	
	\frame{\titlepage}
	
	\begin{frame}{Задача}
		\begin{block}{Постановка}
			Даны записи о произведённых и сломавшихся мечах за 6 месяцев. Сталь для мечей закупалась у двух компаний. Так же известно, что кузнецы делают свою работу идеально.
		\end{block}
		\begin{block}{Цель}
			Необходимо провести разведовательный анализ данных с целью ответа на вопрос: "С каким из поставщиков стали следует заключить договор?"
		\end{block}
	\end{frame}

	\begin{frame}{Кол-во произведённого оружия. График.}
		Общее количество мечей, изготовленных из стали каждой компании.
		\begin{center}
			\includegraphics[width=0.7\textwidth, height=0.7\textheight, 				keepaspectratio]{a.png}
		\end{center}
	\end{frame}	

	\begin{frame}{Кол-во произведённого оружия. Вывод.}
		Из полученного графика видно, что количество произведённых мечей не сильно различается для двух данных видов стали.
		Следовательно, результаты, полученные в ходе анализа будут корректными.
	\end{frame}
	
	\begin{frame}{Распределение дефектов. График.}
		Распределение кол-ва мечей с дефектами для обеих компаний.
		\begin{center}
			\includegraphics[width=\textwidth, height=\textheight, 				keepaspectratio]{b.png}
		\end{center}
	\end{frame}		
	
	\begin{frame}{Распределение дефектов. Вывод.}
		\begin{block}{Harpy.co}
			Много месяцев, где сломалось мало мечей. \\
			И есть месяца, где поломок более 15.
		\end{block}
		\begin{block}{Westeros.inc}
			Нет месяцев, когда ломалось более 15 мечей. \\
			Но есть много месяцев с поломками 4-12 мечей.
		\end{block}
		\begin{block}{Вывод}
			По данным графикам сложно сказать, какая из компаний продаёт более качественную сталь.
		\end{block}
	\end{frame}

	\begin{frame}{Сломанные мечи. График.}
	Взвешенное среднее сломанных мечей по месяцам поломки.
		\begin{center}
		\includegraphics[width=\textwidth, height=\textheight, 				keepaspectratio]{c.png}
		\end{center}
	\end{frame}

	\begin{frame}{Сломанные мечи. График.}
	Поскольку в нашем ряду распределения значения количества сломанных мечей встречаются не с одинаковой частотой, то мы считаем взвешенное среднее:
	\[\overline{x} = \frac{\sum{x_i f_i}}{\sum{f_i}},\]
	где $x_i$ --- это значения, а $f_i$ --- частота этих значений. \\
	
	На диаграмме размаха отражены верхние и нижние границы, квартили,среднее значение, а так же выбросы.
	\end{frame}

	\begin{frame}{Сломанные мечи. Вывод.}
		\begin{block}{Harpy.co}
			За первые три месяца после изготовления очень мало поломок.
			Далее кол-во сломанных мечей резко возрастает.
		\end{block}
		\begin{block}{Westeros.inc}
			Больше всего мечей ломается в первый месяц.
			Далее кол-во поломок постоянно уменьшается.
		\end{block}
	\end{frame}

	\begin{frame}{Сломанные мечи. Среднее. График.}
	Среднее сломанных мечей по месяцам поломки.
		\begin{center}
			\includegraphics[width=\textwidth, height=\textheight, 					keepaspectratio]{d.png}
		\end{center}
	\end{frame}

	\begin{frame}{Вывод.}
		\begin{block}{}
			По результатам анализа можно сказать, что выгоднее покупать сталь у Westeros.inc, так как с течением времени после производства кол-во брака будет только уменьшаться. И даже самое максимальное кол-во брака, приходящееся на первый месяц, всё равно будет меньше, чем возможное кол-во брака у Harpy.co после 3-го месяца.
		\end{block}
		\begin{block}{}
			Но, если будет известно, что продолжительность войны будет не более 3-ёх месяцев, то выгоднее заключить контракт с Harpy.co.
		\end{block}
	\end{frame}

\end{document}